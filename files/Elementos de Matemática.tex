\documentclass[12pt]{book}
\usepackage{xcolor}
\usepackage{amssymb,amsfonts,amstext,amsmath, amsthm}

\setcounter{page}{1}

\usepackage{graphicx,geometry}
\tolerance=8000
\topmargin=-5mm
\textheight=220mm
\textwidth=150mm

\baselineskip=21pt
\parskip=.1in
\parindent=0mm

\geometry{verbose,a4paper,tmargin=3cm,bmargin=3cm,%%%%
	lmargin=3cm,rmargin=3cm,headsep=5mm,footskip=1cm}
\usepackage{float}
\usepackage[brazilian]{babel}
\usepackage[utf8]{inputenc}
\usepackage[T1]{fontenc}
\usepackage{tikz}
\usetikzlibrary{calc}
\usepackage{tocbibind} % Para incluir índice no documento
\usepackage{tikz}
\usepackage{amsmath, amssymb}
\newtheorem{theorem}{Theorem}[section]
\newtheorem{lemma}{Lemma}[section]
\newtheorem{definition}{Definition}[section]
\newtheorem{corollary}{Corollary}[section]
\newtheorem{proposition}{Proposition}[section]
\newtheorem{remark}{Remark}[section]
\newtheorem{example}{Example}[section]
\newtheorem{te}{Teorema}[section]
\newtheorem{supo}{Suposi\c c\~ao}[section]
\newtheorem{algo}{Algoritmo}[section]
\newtheorem{de}{Defini\c c\~ao}[section]
\newtheorem{pr}{Proposi\c c\~ao}[section] 
\newtheorem{lema}{Lema}[section] 
\newtheorem{obs}{Observa\c c\~ao}[section]
\newtheorem{af}{Afirma\c c\~ao}[section]
\newtheorem{ex}{Exemplo}[section]
\newtheorem{prob}{Problema}[section]
\newtheorem{exer}{Exerc\'icio}[section]
\newtheorem{cor}{Corol\'ario}[section]

\newcommand{\dsum}{\displaystyle\sum}
\newcommand{\dlim}{\displaystyle\lim}
\newcommand{\dmin}{\displaystyle\min}
\newcommand{\dmax}{\displaystyle\max}
\newcommand{\ri}{\Rightarrow}
\newcommand{\ov}{\overline}
\newcommand{\df}{\Rightarrow}
\newcommand{\bba}{\begin{eqnarray*}}
	\newcommand{\eea}{\end{eqnarray*}}
\newcommand{\soll}{\stackrel{!}{=}}
\newcommand{\IR}{I\hspace*{-0.15cm}R}
\newcommand{\ID}{I\hspace*{-0.15cm}D}
\newcommand{\IZ}{\slash\!\!\!Z}
\newcommand{\U}{\underline}
\newcommand{\dst}{\displaystyle}
\newcommand{\I}{{\cal I}}
\newcommand{\J}{{\cal J}}
\newcommand{\A}{{\cal A}}
\newcommand{\calAt}{{\tilde{\A}}}
\newcommand{\At}{{\tilde{A}}}
\newcommand{\gf}{\nabla f}
\newcommand{\T}{{\cal T}}
\newcommand{\F}{{\cal F}}
\newcommand{\NN}{{\cal N}}
\newcommand{\tI}{{\tilde I}}
\newcommand{\tJ}{{\tilde J}}
\newcommand{\defeq}{\stackrel{def}{=}}
\newcommand{\R}{I\!\!R}
\newcommand{\N}{I\!\!N}
\def\R{\mathbb{R}}
%\newcommand{\N}{\mathbb{N}}
\def\Z{\mathbb{Z}}
\def\Q{\mathbb{Q}}
\def\I{\mathbb{I}}


\begin{document}
%%%%%%%%%%%%%%%%%%%%%%%%%%%%%%%%%%%%%%%%%%%%%%%%%%%%%%%%%%%%%
{\Large
	
%\begin{figure}[h]
%	\centering\includegraphics[scale=0.52]{opim.png}
	
%\end{figure}
\vspace{4.5cm}
{\sc \Huge
	\centerline{Elementos de Matemática}

	\vspace{7.5cm}}
	
		
}

{\Large \begin{flushright}
	\begin{minipage}[t]{7.0cm}
		Vitaliano S. Amaral - UFPI
	\end{minipage}
\end{flushright}
}
%\institute{...}

\date{\today}
% The correct dates will be entered by the editor


%\maketitle
\tableofcontents

%
% \noindent{\bf AMS Classification.} 90C29\,$\cdot$\,90C25\,$\cdot$\,65K10\,$\cdot$\,49J52.
%
\chapter{Números Reais}

Ao longo da história, os números surgiram para atender a diferentes necessidades humanas. Os números naturais apareceram inicialmente como uma forma de contar objetos e registrar quantidades. Com o tempo, a necessidade de representar dívidas, perdas e posições relativas levou à introdução dos números inteiros, que incluem tanto os naturais quanto seus opostos negativos.


O conjunto dos números naturais é dado por:
$$
\mathbb{N} = \{0, 1, 2, \dots\}
$$

Cada número natural pode ser representado sobre a reta real. Uma vez fixados os pontos correspondentes aos números $0$ e $1$, adotamos a distância entre eles como unidade de medida. A partir daí, todos os números naturais são representados como pontos igualmente espaçados, posicionados da esquerda para a direita a partir do zero.

Veja a seguir a ilustração do conjunto $\mathbb{N}$ sobre a reta real:

\begin{center}
	\begin{tikzpicture}[scale=1]
		% Reta para N
		\draw[->] (-0.5,0) -- (10.5,0);
		
		% Marcas e números
		\foreach \x in {0,...,10}
		{
			\draw (\x,0.1) -- (\x,-0.1);
			\node[below] at (\x,-0.1) {\x};
		}
	\end{tikzpicture}
\end{center}

O conjunto dos números inteiros é representado da seguinte forma:
$$
\mathbb{Z} = \{\dots, -2, -1, 0, 1, 2, \dots\}
$$

De maneira análoga, os números inteiros também podem ser representados sobre a reta real. Uma vez fixados os pontos $0$ e $1$, adotamos a distância entre eles como unidade de medida. A partir disso, os inteiros são posicionados igualmente espaçados, estendendo-se para a direita (inteiros positivos) e para a esquerda (inteiros negativos) a partir do zero.

Veja a seguir a ilustração do conjunto $\mathbb{Z}$ sobre a reta real:

\begin{center}
	\begin{tikzpicture}[scale=1]
		% Reta para Z
		\draw[->] (-5.5,0) -- (5.5,0);
		
		% Marcas e números
		\foreach \x in {-5,...,5}
		{
			\draw (\x,0.1) -- (\x,-0.1);
			\node[below] at (\x,-0.1) {\x};
		}
	\end{tikzpicture}
\end{center}

Observamos que os números inteiros consecutivos delimitam intervalos unitários (de comprimento $1$).

O conjunto dos números racionais surge da necessidade de representar partes de um inteiro, aparecendo como subdivisões desses intervalos unitários. Por exemplo, $\frac{1}{2}$ corresponde ao ponto situado exatamente no meio entre $0$ e $1$, $\frac{3}{4}$ está localizado a três quartos da distância entre $0$ e $1$, e assim por diante. Os racionais negativos seguem a mesma lógica, mas posicionados à esquerda de $0$.

Assim, o conjunto dos números racionais é representado por:
$$
\mathbb{Q} = \left\{ \dfrac{p}{q} \;|\; p, q \in \mathbb{Z},\, q \neq 0 \right\}
$$

Essa construção nos permite associar um ponto da reta real a cada número racional. No entanto, como entre quaisquer dois números reais distintos existem infinitos racionais, não podemos representá-los todos graficamente. Em vez disso, destacamos apenas alguns exemplos para ilustrar a densidade dos números racionais sobre a reta real.

\begin{center}
	\begin{tikzpicture}[scale=1]
		% Reta
		\draw[->] (-5.5,0) -- (5.5,0);
		
		% Inteiros com rótulo
		\foreach \x in {-5,0,2,5} {
			\draw (\x,0.1) -- (\x,-0.1);
			\node[below] at (\x,-0.1) {\x};
		}
		
		% Inteiros sem rótulo (apenas traço)
		\foreach \x in {-4,-3,-2,-1,1,3,4} {
			\draw (\x,0.05) -- (\x,-0.05);
		}
		
		% Racionais destacados corretamente posicionados
		\draw (-3.5,0.1) -- (-3.5,-0.1);
		\node[below] at (-3.5,-0.1) {$-\tfrac{7}{2}$};
		
		\draw (-1.5,0.1) -- (-1.5,-0.1);
		\node[below] at (-1.5,-0.1) {$-\tfrac{3}{2}$};
		
		\draw (-0.5,0.1) -- (-0.5,-0.1);
		\node[below] at (-0.5,-0.1) {$-\tfrac{1}{2}$};
		
		\draw (0.75,0.1) -- (0.75,-0.1);
		\node[below] at (0.75,-0.1) {$\tfrac{3}{4}$};
		
		\draw (1.5,0.1) -- (1.5,-0.1);
		\node[below] at (1.5,-0.1) {$\tfrac{3}{2}$};
		
		\draw (2.5,0.1) -- (2.5,-0.1);
		\node[below] at (2.5,-0.1) {$\tfrac{5}{2}$};
		
		\draw (3.5,0.1) -- (3.5,-0.1);
		\node[below] at (3.5,-0.1) {$\tfrac{7}{2}$};
	\end{tikzpicture}
\end{center}

Admitiremos as seguintes operações (adição e multiplicação) no conjunto dos números racionais.


\begin{de}{\bf(Adição)} Sejam $a=\displaystyle\frac{m}{n}$ e $b=\displaystyle\frac{r}{s}$ elementos de $\Q$. A soma de $a$ com $b$ é o elemento de $\Q$
	$$a+b=\displaystyle\frac{ms+nr}{ns}.$$
\end{de}
\begin{ex}
	Sejam $a = \displaystyle\frac{2}{3}$ e $b = \displaystyle\frac{5}{4}$. Então, a soma de $a$ com $b$ é:
	\[
	a + b = \frac{2 \cdot 4 + 3 \cdot 5}{3 \cdot 4} = \frac{8 + 15}{12} = \frac{23}{12}.
	\]
\end{ex}


\begin{de}{\bf(Multiplicação)} Sejam $a=\displaystyle\frac{m}{n}$ e $b=\displaystyle\frac{r}{s}$ elementos de $\Q$. A multiplicação(produto) de $a$ com $b$ é o elemento de $\Q$
	$$ab=\displaystyle\frac{mr}{ns}.$$
\end{de}

\begin{ex}
	Sejam $a = \displaystyle\frac{2}{3}$ e $b = \displaystyle\frac{5}{4}$. Então, o produto de $a$ com $b$ é:
	\[
	ab = \frac{2 \cdot 5}{3 \cdot 4} = \frac{10}{12} = \frac{5}{6}.
	\]
\end{ex}
É fácil perceber que entre dois números racionais sempre existe outro número racional. De fato, dados dois números racionais $a=\dfrac{m}{n}$ e $b=\dfrac{r}{s}$ com $a < b$, $m,n,r,s\in\Z$. Considere o número racional da forma 
\[
c=a+\frac{b-a}{2}.
\]
Podemos observar que $c$ está entre $a$ e $b$, pois $c$ é obtido somando a $a$ a metade da distância entre $a$ e $b$. Veja a ilustração geométrica na Figura~\ref{fig:meio}.
\begin{figure}[H]
	\centering
	\begin{tikzpicture}[scale=8]
		% Eixo
		\draw[->] (0,0) -- (1.1,0) node[right] {};
		
		% Pontos
		\filldraw (0.2,0) circle (0.3pt) node[below] {$a$};
		\filldraw (0.7,0) circle (0.3pt) node[below] {$b$};
		\filldraw (0.45,0) circle (0.3pt) node[below] {$c$};
		
		% Marcação de distância
		%\draw[<->] (0.2,0.1) -- (0.7,0.1) node[midway, above] {$b - a$};
		\draw[dashed] (0.45,0) -- (0.45,0.08);
		\draw[dashed] (0.2,0) -- (0.2,0.08);
		\draw[<->] (0.2,0.08) -- (0.45,0.08) node[midway, above] {$\frac{b-a}{2}$};
	\end{tikzpicture}
	\caption{Ilustração do ponto médio $c=a+\frac{b-a}{2} $  na reta real.}
	\label{fig:meio}
\end{figure}

Agora, além da afirmação acima vamos mosttrar que $c$ é um número racional e está enrte os racionasi $a$ e $b$. Veja que 
\begin{eqnarray*}
	c=a+\frac{1}{2}(b-a)&=&\frac{2a+b-a}{2}=\frac{a + b}{2}\\
	&=&\frac{1}{2}\left(\dfrac{r}{s}+\dfrac{m}{n}\right)=\left(\dfrac{rn+ms}{2ns}\right),
\end{eqnarray*}
como $rn + ms$ e $2ns$ são números inteiros, podemos garantir que $c$ é um número racional, pois é a razão entre dois inteiros com denominador diferente de zero.

Além da explicação anterior, outra forma de garantir que $c$ está entre $a$ e $b$ é observar que
\[
a = \frac{a + a}{2} < \frac{a + b}{2} = c \quad \text{e} \quad c = \frac{a + b}{2} < \frac{b + b}{2} = b,
\]
portanto, temos $a < c < b$.

Diante do exposto anteriormente, surge uma dúvida: como sempre existe um número racional entre dois números racionais, então seria possível preencher toda a reta numérica apenas com números racionais?

A seguir veremos que a resposta para a pergunta anterior é: não é possível. 

Diz-se que Hipaso de Metaponto, um seguidor de Pitágoras, foi o primeiro a descobrir que existem números que não podem ser representados pela divisão de dois números inteiros. Ele teria demonstrado que $\sqrt{2}$ não é racional, provavelmente por meio de uma prova geométrica.


Considere um triângulo retângulo desenhado sobre a reta numérica(veja Figura \ref{fig:h}), com catetos medindo $1$ unidade cada e hipotenusa sobre a reta numérica, indo do ponto $0$ até o ponto marcado por $x$, ou seja, a hipotenusa tem comprimento medindo $x$ unidades.

\begin{center}
	\begin{figure}[h]
		\centering
		\begin{tikzpicture}[scale=4]
			
			% valor da raiz de 2
			\pgfmathsetmacro{\rtwo}{sqrt(2)}
			
			% eixo da reta real
			\draw[->] (-0.5,0) -- (2.2,0) node[right] {$\mathbb{R}$};
			
			% marcas inteiras
			\foreach \x in {0,1,2} {
				\draw (\x,0.05) -- (\x,-0.05) node[below] {\x};
			}
			
			% pontos da hipotenusa sobre a reta
			\coordinate (O) at (0,0);            
			\coordinate (H) at (\rtwo,0);        
			\coordinate (V) at ({\rtwo/2},{sqrt(1 - (\rtwo/2)^2)}); 
			
			% triângulo
			\draw[thick] (O) -- (V) -- (H) -- cycle;
			
			% indicação dos catetos
			\node[left]  at ($(O)!0.5!(V)$) {$1$};
			\node[right] at ($(V)!0.5!(H)$) {$1$};
			
			% marca ângulo reto com quadradinho
			\path (V) -- ($(V)!0.15!(O)$) coordinate (Va);
			\path (V) -- ($(V)!0.15!(H)$) coordinate (Vb);
			\draw (Va) -- ($(Va)+(Vb)-(V)$) -- (Vb);
			
			% marca x minúsculo abaixo do ponto final da hipotenusa
			\draw (\rtwo,0.05) -- (\rtwo,-0.05) node[below] {$x$};
			
		\end{tikzpicture}
		\caption{}
		\label{fig:h}
	\end{figure}	
\end{center}



Pelo \textbf{Teorema de Pitágoras}, temos:
$
x^2 = 1^2 + 1^2 = 2.$

Suponha, por contradição, que $x$ seja um número racional. Então podemos escrevê-lo como $\dfrac{p}{q}$, com $p$ e $q$ inteiros primos entre si. Substituindo em $x^2 = 2$:
\[
\left(\frac{p}{q}\right)^2 = 2 
\quad\Rightarrow\quad 
\frac{p^2}{q^2} = 2
\]
\[
p^2 = 2q^2
\]
Isso implica que $p^2$ é par, logo $p$ é par. Seja $p = 2k$, assim temos
\[
(2k)^2 = 2q^2 
\quad\Rightarrow\quad 
4k^2 = 2q^2 
\quad\Rightarrow\quad 
q^2 = 2k^2
\]
Portanto, $q^2$ também é par, o que implica que $q$ é par.

Chegamos a uma contradição, pois $p$ e $q$ seriam ambos pares, contrariando a hipótese de que são primos entre si. Logo, $x$ \textbf{não é um número racional}.

Como $x$ é um ponto da reta numérica que não pertence ao conjunto dos racionais, concluímos que a reta real não pode ser preenchida completamente apenas por números racionais.

O conjunto dos números que não podem ser representados como a divisão de dois inteiros, ou seja, que não são números racionais, é denotado pela letra $\mathbb{I}$ e chamado de \emph{conjunto dos números irracionais}.


Da própria definição, temos que os conjuntos $\mathbb{Q}$ e $\mathbb{I}$ não possuem elementos em comum, isto é,
\[
\mathbb{Q} \cap \mathbb{I} = \emptyset.
\]

Os elementos dos conjuntos $\mathbb{Q}$ e $\mathbb{I}$, juntos, formam o conjunto dos números reais, denotado por $\mathbb{R}$, ou seja,
\[
\mathbb{R} = \mathbb{Q} \cup \mathbb{I}.
\]


\begin{exer}
	Mostre que a soma de dois números racionais é também um número racional.
\end{exer}
\begin{exer}
	A soma de uma número racional com um número irracional é um número racional? Justifique sua resposta.
\end{exer}
\begin{exer}
	A soma de dois números irracionais é um número irracional? Justifique sua resposta.
\end{exer}




	
	
	
	
	
	
\end{document}
